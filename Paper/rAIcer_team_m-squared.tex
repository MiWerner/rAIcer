\documentclass[11pt,final,journal,a4paper,towside,towcolumn]{IEEEtran}
\usepackage{cite}
\usepackage{graphicx}
\usepackage[utf8]{inputenc}
%set headings
%\markboth{Werner \MakeLowercase{Inverse kinematics for high redundant robots with 7 DOF}{Werner \MakeLowercase{Inverse kinematics for high redundant robots with 7 DOF}

\begin{document}
%Set title and author
\title{Abschlussbericht des Team $m^2$ zu dem \\Projektpraktikum Robotik und Automation:\\Künstliche Intelligenz}
\author{Marius Krusen und F. J. Michael Werner}
\maketitle

\begin{abstract}
Brauchen wir ein Abstrakt?
\end{abstract}

\section{Aufgabenstellung}
\IEEEPARstart{A}{t} erster Buchstabe muss fett und seeeehr groß sein.
\\

\section{Lösungsansatz}

\subsection{NEAT}
NEAT wurde in \cite{stanley:gecco02-efficient} vorgestellt
\subsection{Streckenerkennung}

\subsection{Featureberechnung}

\section{Ergebnisse}

\section{Zusammenfassung}

\bibliography{Quellen}{}
\bibliographystyle{./IEEEtranBST2/IEEEtran}
\end{document}